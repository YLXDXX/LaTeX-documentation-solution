\section{基本环境}

\subsection{\TeX Live 2020 安装}

下载 \TeX Live 2020套件的镜像,可以在各个镜像站中直接下载,例如:
\begin{itemize}
	\item 
	清华镜像:
	\url{https://mirrors.tuna.tsinghua.edu.cn/CTAN/systems/texlive/Images/}
	\item 
	阿里云镜像:
	\url{https://mirrors.aliyun.com/CTAN/systems/texlive/Images/}
\end{itemize}

下载好后挂载镜像文件,到挂载的目录执行 \lstinline!sudo ./install-tl! 即可。安装好后,默认目录在 \lstinline|/usr/local/texlive/2020/| ,可执行程序在 \lstinline|bin| 目录里。这里不建议修改环境变量,后面需要使用 \TeX 时直接指定目录即可。有一些程序运行依赖 \TeX ,安装的时候直接从源里面安装即可,源里面安装的 \TeX 与我们自己安装的 \TeX 互不干扰。 

安装完成过后,由于 \TeX Live 2020 有一些宏包没有收录进去,需要我们手动安装,以 pgfornament 宏包为例,从 CTAN 下载后解压,然后执行如下命令:
\begin{lstlisting}[language=bash]
mkdir -p /usr/local/texlive/2020/texmf-dist/doc/latex/pgfornament
mkdir -p /usr/local/texlive/2020/texmf-dist/tex/latex/pgfornament
mkdir -p /usr/local/texlive/2020/texmf-dist/tex/generic/pgfornament
cp -r doc/* /usr/local/texlive/2020/texmf-dist/doc/latex/pgfornament/
cp -r latex/* /usr/local/texlive/2020/texmf-dist/tex/latex/pgfornament/
cp -r generic/* /usr/local/texlive/2020/texmf-dist/tex/generic/pgfornament/
/usr/local/texlive/2020/bin/x86_64-linux/mktexlsr
\end{lstlisting}
上面最后一句命令是刷新  \TeX Live 的缓存,让其识别我们手动安装的宏包。\footnote{若没有刷新\TeX Live,即使我们安装成功,在调用的时候依就会报缺少该宏包的错误。}

\subsection{TeXStudio 编辑器}

后端的环境安装好后,我们前端的编辑器就该登场了。这里推荐 TeXStudio ,基本功能应有尽有还带有强大的宏(Macros)。之前一位印度小哥分享了他在数学课上使用\LaTeX  \ and Vim 飞速记笔记的\hyperref{https://castel.dev/post/lecture-notes-1/}{category}{name}{文章} 。他速度的提升得益于 vim 中的 snippets 功能插件,然而在 TeXStudio 中,我们借助其中的宏,完全可以实现snippets的功能,TeXStudio 的宏可使用 javascript 语言编写,触发条件为正则表达式,也可以自定义快捷键,更为强大。唯一的缺点是没法像vim那样使用数字定义优先级,当宏的数量太多时,调整起来较为麻烦。


编译器推荐设置为 xelatex ,对中文更加友好。对应的设置命令为:
\begin{lstlisting}[language=bash]
"/usr/local/texlive/2020/bin/x86_64-linux/xelatex" -synctex=1 -interaction=nonstopmode --shell-escape %.tex
\end{lstlisting}

设置好后,可以在TeXStudio里面跑一个例子,file $ \rightarrow $ New From Template 选择其中一个模板,编译一下,再调用  \lstinline|\usepackage{ctex}|  宏包,在文档里面输入中文,再编译一下,若没有问题则基本环境准配置就完成了。





